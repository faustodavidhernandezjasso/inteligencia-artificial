\documentclass[a4paper]{article} 
\input{head}
\newcommand{\pow}[2]{#1^{#2}}
\newcommand{\supra}[1]{\textsuperscript{#1}}
\begin{document}

%-------------------------------
%	TITLE SECTION
%-------------------------------

\fancyhead[C]{}
\hrule \medskip % Upper rule
\begin{minipage}{0.295\textwidth} 
\raggedright
\footnotesize
Fausto David Hernández Jasso \hfill\\   
@FaustoJH \hfill\\
fausto.david.hernandez.jasso@ciencias.unam.mx
\end{minipage}
\begin{minipage}{0.4\textwidth} 
\centering 
\large 
Inteligencia Artificial\\ 
\normalsize 
Introducción a \texttt{Python} \\ 
\end{minipage}
\begin{minipage}{0.295\textwidth} 
\raggedleft
\today\hfill\\
\end{minipage}
\medskip\hrule 
\bigskip
\section{¿Qué es \texttt{Python}?}
\noindent
Es un lenguaje de programación potente y rápido, se lleva bien con todos, corre en cualquier plataforma
es amistoso y sencillo de aprender.
\section{Ejecutar un programa en \texttt{Python}}
Supongamos que un archivo llamado \texttt{hello\_world.py} tenemos el siguiente programa en \texttt{Python}
\begin{lstlisting}[language=Python]
    print('Hello World!')
\end{lstlisting}
Se ejecuta de la siguiente manera:
\begin{lstlisting}[language=bash]
    $ python hello\_world.py
\end{lstlisting}
\section{Variables}
\noindent
Podemos \textbf{asociar} nombres a cualquier tipo de objeto usando el operador de asignación 
\texttt{=}
\newline 
\(\langle \text{\texttt{name}} \rangle = \langle \text{\texttt{value}} \rangle\)
\newline 
Una variable siempre puede ser reasignada a diferentes valores de diferentes tipos durante su
ciclo de vida.
\begin{lstlisting}[language=Python]
    var = 1
    print(var) # Imprime el 1
    var = 2
    print(var) # Imprime el 2
    print(type(var)) # Nos dice que el tipo de la variable es un entero
    var = 'Hola'
    print(var) # Imprime la cadena Hola
    print(type(var)) # Nos dice que el tipo de la variable es un string
    var = [1, 2, 3]
    print(type(var)) # Nos dice que el tipo de la variable es una lista
\end{lstlisting}
\subsubsection{Constantes}
\noindent
Las constantes son nombres a un valor que solamente son asignados una vez durante todo el programa,
es decir, siempre vale lo mismo.
\begin{lstlisting}[language=Python]
    CONSTANTE_DE_GRAVEDAD = 9.81
    PI = 3.1415
    MI_CONSTANTE = 3212
\end{lstlisting}

\end{document}
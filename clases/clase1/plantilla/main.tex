\documentclass[a4paper]{article} 
\addtolength{\hoffset}{-2.25cm}
\addtolength{\textwidth}{4.5cm}
\addtolength{\voffset}{-3.25cm}
\addtolength{\textheight}{5cm}
\setlength{\parskip}{0pt}
\setlength{\parindent}{0in}

%----------------------------------------------------------------------------------------
%	PACKAGES AND OTHER DOCUMENT CONFIGURATIONS
%----------------------------------------------------------------------------------------
\usepackage{mathtools} % Package for using math tools
\usepackage[inline]{enumitem} % Enumerate environment
\usepackage{blindtext} % Package to generate dummy text
\usepackage{charter} % Use the Charter font
\usepackage[utf8]{inputenc} % Use UTF-8 encoding
\usepackage{microtype} % Slightly tweak font spacing for aesthetics
\usepackage[spanish]{babel} % Language hyphenation and typographical rules
\usepackage{amsthm, amsmath, amssymb} % Mathematical typesetting
\usepackage{float} % Improved interface for floating objects
\usepackage[final, colorlinks = true, 
            linkcolor = black, 
            citecolor = black]{hyperref} % For hyperlinks in the PDF
\usepackage{graphicx, multicol} % Enhanced support for graphics
\usepackage{xcolor} % Driver-independent color extensions
\usepackage{marvosym, wasysym} % More symbols
\usepackage{rotating} % Rotation tools
\usepackage{censor} % Facilities for controlling restricted text
\usepackage{listings, style/lstlisting} % Environment for non-formatted code, !uses style file!
\usepackage{pseudocode} % Environment for specifying algorithms in a natural way
\usepackage{style/avm} % Environment for f-structures, !uses style file!
\usepackage{booktabs} % Enhances quality of tables
\usepackage{tikz-qtree} % Easy tree drawing tool
\tikzset{every tree node/.style={align=center,anchor=north},
         level distance=2cm} % Configuration for q-trees
\usepackage{style/btree} % Configuration for b-trees and b+-trees, !uses style file!
\usepackage[backend=biber,style=numeric,
            sorting=nyt]{biblatex} % Complete reimplementation of bibliographic facilities
\addbibresource{ecl.bib}
\usepackage{csquotes} % Context sensitive quotation facilities
\usepackage[yyyymmdd]{datetime} % Uses YEAR-MONTH-DAY format for dates
\renewcommand{\dateseparator}{-} % Sets dateseparator to '-'
\usepackage{fancyhdr} % Headers and footers
\pagestyle{fancy} % All pages have headers and footers
\fancyhead{}\renewcommand{\headrulewidth}{0pt} % Blank out the default header
\fancyfoot[L]{} % Custom footer text
\fancyfoot[C]{} % Custom footer text
\fancyfoot[R]{\thepage} % Custom footer text
\newcommand{\note}[1]{\marginpar{\scriptsize \textcolor{red}{#1}}} % Enables comments in red on margin


%----------------------------------------------------------------------------------------

\newcommand{\pow}[2]{#1^{#2}}
\newcommand{\supra}[1]{\textsuperscript{#1}}
\begin{document}

%-------------------------------
%	TITLE SECTION
%-------------------------------

\fancyhead[C]{}
\hrule \medskip % Upper rule
\begin{minipage}{0.295\textwidth} 
\raggedright
\footnotesize
Fausto David Hernández Jasso \hfill\\   
@FaustoJH \hfill\\
fausto.david.hernandez.jasso@ciencias.unam.mx
\end{minipage}
\begin{minipage}{0.4\textwidth} 
\centering 
\large 
Inteligencia Artificial\\ 
\normalsize 
Introducción a \texttt{Python} \\ 
\end{minipage}
\begin{minipage}{0.295\textwidth} 
\raggedleft
\today\hfill\\
\end{minipage}
\medskip\hrule 
\bigskip
\section{¿Qué es \texttt{Python}?}
\noindent
Es un lenguaje de programación potente y rápido, se lleva bien con todos, corre en cualquier plataforma
es amistoso y sencillo de aprender.
\section{Ejecutar un programa en \texttt{Python}}
Supongamos que un archivo llamado \texttt{hello\_world.py} tenemos el siguiente programa en \texttt{Python}
\begin{lstlisting}[language=Python]
    print('Hello World!')
\end{lstlisting}
Se ejecuta de la siguiente manera:
\begin{lstlisting}[language=bash]
    $ python hello\_world.py
\end{lstlisting}
\section{Variables}
\noindent
Podemos \textbf{asociar} nombres a cualquier tipo de objeto usando el operador de asignación 
\texttt{=}
\newline 
\(\langle \text{\texttt{name}} \rangle = \langle \text{\texttt{value}} \rangle\)
\newline 
Una variable siempre puede ser reasignada a diferentes valores de diferentes tipos durante su
ciclo de vida.
\begin{lstlisting}[language=Python]
    var = 1
    print(var) # Imprime el 1
    var = 2
    print(var) # Imprime el 2
    print(type(var)) # Nos dice que el tipo de la variable es un entero
    var = 'Hola'
    print(var) # Imprime la cadena Hola
    print(type(var)) # Nos dice que el tipo de la variable es un string
    var = [1, 2, 3]
    print(type(var)) # Nos dice que el tipo de la variable es una lista
\end{lstlisting}
\subsubsection{Constantes}
\noindent
Las constantes son nombres a un valor que solamente son asignados una vez durante todo el programa,
es decir, siempre vale lo mismo.
\begin{lstlisting}[language=Python]
    CONSTANTE_DE_GRAVEDAD = 9.81
    PI = 3.1415
    MI_CONSTANTE = 3212
\end{lstlisting}

\end{document}